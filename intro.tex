\chapter{Introdução}
A utilização de métodos computacionais na Geografia vem se desenvolvendo durante esta última década de uma forma nunca vista na história. Esta evolução não se restringe apenas a capacidade computacional de processar dados de forma mais rápida, mas também na possibilidade de processá-los em sí, além da possibilidade de armazenar uma quantidade surpreendentemente grande de dados. Desde a consolidação do pensamento quantitativo e analítico na Geografia, assim como durante suas revoluções paradigmáticas, modelos, métodos, teorias e ferramentas foram desenvolvidas para se aproveitar das possibilidades, e também dos novos questionamentos em que o novo campo da computação traria ao mundo. Muitas das ideias foram sendo então desenvolvidas, integradas e popularizadas ao longo da história de acordo com a evolução tecnológica e a capacidade de entender digitalmente o espaço em suas muitas dimensões. \par
Vivemos uma época em que o intelectual humano se caracteriza cada vez mais como digital. O pensar crítico se mistura ao quantificar das incertezas promovida por uma busca  muitas vezes frustada de transdisciplinaridade, gerando técnicas inacabadas de análise dos espaços em uma busca constante ao entendimento da sua enorme complexidade. O entendimento kuhniano do fazer científico começa a se curvar a uma Geografia em que não consegue mais lidar com o fato de ignorar que novas possibilidades do fazer geográfico de fato surgiram, ignorando muitos dos paradigmas em vigor. O processo de armazenamento e análise do dado geográfico pode sim se juntar a buscas mais qualificativas. O Sistema de Informação Geográfico (SIG) pode ser crítico, assim como o uso de geotecnologias e métodos quantitativos pode ser sim utilizado por áreas de conhecimento tipicamente de "humanas". \par
Este fenômeno vem acontecendo e crescendo de acordo com o desenvolvimento de novas ferramentas que possibilitam que a implementação do modelo ou do método possam ser realizadas não só por especialistas da área da computação, mas principalmente por especialistas das próprias áreas de aplicação da técnica em questão. Isso poupa gastos e principalmente favorece que tal etapa de implementação seja realizada da melhor forma possível, já que poderá evitar possíveis desentendimentos conceituais entre os profissionais envolvidos. Além disso, muitas técnicas mais antigas como as ligadas a área de banco de dados vem se desenvolvendo neste mesmo sentido afim de alcançar estes mesmos profissionais e popularizar formas de se lidar com a técnica de uma nova maneira. \par
No entanto, pela evolução ter se dado de forma repentina, muitas ideias e conceitos devem ser melhor estudados para suprir tais aplicações. Grande parte deste problema conceitual se dá pelo fato de que a maioria desses métodos computacionais derivarem de ambientes puramente lógicos, onde suas ontologias já estão de fato mais consolidadas. Ao tentar se aplicar tais técnicas a um ambiente computacional espacial, tais conceitos passam a não servir como antes. O processo de abstração necessário para a transformação da informação espacial para um ambiente computacional é um desafio muito maior do que o normal, já que tal processo envolve a implementação de algoritmos que entendam hierarquicamente o espaço analisado\cite{CAMARA_etal04}. O computador necessita então aplicar tais processamentos de forma que siga não somente uma lógica clássica como também topológica. \par
Outro ponto importante nesta evolução é a possibilidade de distribuição e popularização de dados espaciais na web transformar as discussões realizadas na academia mais abertas ao grande público. As novas ferramentas ligadas a geotecnologia vem permitindo que não somente novos tipos de processamento de dados espaciais sejam realizados, como permite também que tais dados sejam distribuídos em larga escala através da internet. Este tipo de processo é importante não só pela própria distribuição de dados e informação, mas também como uma forma de aproximar o conhecimento geográfico a qualquer tipo de pessoa, incluindo os próprios Geógrafos. É um processo de democratização da informação espacial. \par
Neste sentido, o presente trabalho objetiva realizar uma discussão teórico-conceitual e histórica através da história da Computação e da Geografia, encontrando momentos de interseção através do conhecimento epistemológico, buscando entender como tais geotecnologias podem influenciar no modo em que um especialista realiza a transformação de elementos observados na paisagem para variáveis dentro de um ambiente computacional. Para isso, é preciso entender primeiramente quais foram as ideias e os conceitos que mais influenciaram a Geografia durante sua história para que tal ciência passasse a utilizar ferramentas que mudariam a forma de se estudar o espaço, assim como quais foram exatamente estas ferramentas. Será discutido em que ponto está a relação dos Geógrafos com as novas tecnologias e sua consequente interação com a ciência de quarto paradigma\cite{HEY_etal09}. Além disso, objetiva-se entender como se dá, através da visão de um estudo de caso, a transformação do mundo real observável em universo ontológico, formal, estrutural e de implementação aplicando-se a elaboração de um modelo de dados conceitual para o desenvolvimento de um banco de dados geográfico para suporte a estudos de favorabilidade à recuperação florestal. \par
A área do estudo de caso será a bacia hidrográfica do rio São João (BHRSJ), mas o objetivo do trabalho está além desta delimitação espacial. Buscaremos na verdade entender o processo de criação de uma base de dados espaciais, suas tecnologias e características, procurando contribuir para o desenvolvimento de práticas similares. A partir do entendimento das características e das técnicas utilizadas no fazer geográfico atual, e no papel do Geógrafo como um utilizador da tecnologia a seu favor, o trabalho busca desenvolver um software protótipo que possua uma interface de comunicação e análise de dados geográficos a partir do uso do banco de dados criado. Tal interface visa contribuir para a Geografia e seus atores de forma com que o uso progressivo de novas geotecnologias não sejam necessariamente ditadas pelo mercado privado, mas sim construídas de forma aberta, livre e de forma colaborativa, atendendo também necessidades locais. \par
Especificamente objetiva-se: \par
\begin{enumerate}
	\item Realizar uma pesquisa bibliográfica assim como uma discussão teórica-conceitual sobre a evolução das teorias, conceitos e técnicas computacionais e geocomputacionais relacionando-as a história do pensamento geográfico, assim como a relação da Geografia e do Geógrafo como usuários dessas ferramentas.
	\item Realizar o processo de abstração e transformação da realidade para um ambiente computacional implementando um modelo de dados utilizando o método OMT-G no SGBDOR PostgreSQL/PostGIS. Tal implementação é realizada visando a construção de um sistema que  possibilite a instituição ou especialista utilizá-lo como um sistema de apoio a tomada de decisões voltado para estudos de favorabilidade à recuperação florestal e também como uma plataforma de distribuição de dados espaciais na web. O modelo deve seguir como referência os dados gerados pelo estudo desenvolvido por Seabra\cite{SEABRA} por se tratar de um trabalho de referência na área.
	\item Desenvolver um SIG que possua uma interface gráfica simplificada de acesso ao banco de dados geográfico, pensado e desenvolvido especificamente para usuários que possuam pouco conhecimento técnico computacional. A ideia é que tal aplicação sirva para que Geógrafos e pesquisadores de uma forma geral que não possuam uma bagagem quantitativa, também possam interagir com novas geotecnologias derivadas de um novo paradigma do fazer científico, buscando aumentar ainda mais a contribuição de novos saberes e interfaces intelectuais complexas.
\end{enumerate}